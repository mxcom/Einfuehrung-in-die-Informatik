\subsection{Dezimal $\rightarrow$ Binär}

1. Methode:\\

$314_{10} = 256 + 32 + 16 + 8 + 2 = 100111010_2$

\begin{table}[h]
\begin{tabular}{|c|c|c|c|c|c|c|c|c|}
\hline
256 & 128 & 64 & 32 & 16 & 8 & 4 & 2 & 1\\
\hline
1   &  0  &  0 &  1 &  1 & 1 & 0 & 1 & 0\\
\hline
\end{tabular}
\end{table}

2. Methode:\\

\begin{tabular}{cccccccc}
314 & $:$ & 2 & $=$ & 157 & R & 0&\multirow{9}*{\limitarrow}\\
157 & $:$ & 2 & $=$ &  78 & R & 1\\
 78 & $:$ & 2 & $=$ &  39 & R & 0\\
 39 & $:$ & 2 & $=$ &  19 & R & 1\\
 19 & $:$ & 2 & $=$ &   9 & R & 1\\
  9 & $:$ & 2 & $=$ &   4 & R & 1\\
  4 & $:$ & 2 & $=$ &   2 & R & 0\\
  2 & $:$ & 2 & $=$ &   1 & R & 0\\
  1 & $:$ & 2 & $=$ &   0 & R & 1\\
\end{tabular}

\subsection{Dezimal ($\rightarrow$ Binär) $\rightarrow$ Hexadezimal}

1. Methode:\\

$314_{10} = \text{13A}_{16}$\\

\begin{tabular}{cccccccc}
314 & $:$ & 16 & $=$ & 19 & R & 10 & \multirow{3}*{\limitarroww}\\
 19 & $:$ & 16 & $=$ &  1 & R & 3\\
  1 & $:$ & 16 & $=$ &  0 & R & 1\\
\end{tabular}\
\\~\\

2. Methode:\\

- Zuerst Dezimal in Binär umwandeln\\

$314_{10} = 100111010_2$\\

- Binär in Hexadezimal umwandeln\\

$1 0011 1010_2$ in Viererblöcke umschreiben (links ggfs. mit Nullen auffüllen). Die Viererblöcke in ihre entsprechende Dezimaldarstellung umrechnen und diese dann in die entsprechende Hexadezimalddarstellung.

\begin{table}[h]
\begin{tabular}{|r||c|c|c|}
\hline
Binär: & 0001 & 0011 & 1010\\
\hline
Dezimal: & 1 & 3 & 10\\
\hline
Hexadezimal: & 1& 3 & A\\
\hline
\end{tabular}
\end{table}

$100111010_2 = \text{13A}_{16}$

\subsection{Binär $\rightarrow$ Dezimal}

\begin{align*}
100111010_2 &= 2^{8} \cdot 1 + 2^{7} \cdot 0 + 2^{6} \cdot 0 + 2^{5} \cdot 1 + 2^{4} \cdot 1 + 2^{3} \cdot 1 + 2^{2} \cdot 0 + 2^{1} \cdot 1 + 2^{0} \cdot 0\\
&= 256 + 32 + 16 + 8 + 2\\
&= 314_{10}
\end{align*}

\subsection{Hexadezimal $\rightarrow$ Dezimal}

\begin{align*}
\text{13A}_{16} &= 16^2 \cdot 1 + 16^1 \cdot 3 + 16^0 \cdot 10\\
&= 256 + 48 + 10\\
&= 314_{10}
\end{align*}



