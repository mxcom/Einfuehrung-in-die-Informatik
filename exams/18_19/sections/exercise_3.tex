\section*{Aufgabe 3 (10 Punkte)}

In einem 32-bit Wort $w$ sind drei Bitfelder $a$, $b$ und $c$ folgendermaßen definiert:

\begin{table}[h]
\centering
\resizebox{\textwidth}{!}{\begin{tabular}{|c|c|c|c|c|c|c|c|c|c|c|c|c|c|c|c|c|c|c|c|c|c|c|c|c|c|c|c|c|c|c|c|}
\hline
31 & 30 & 29 & 28 & 27 & 26 & 25 & 24 & 23 & 22 & 21 & 20 & 19 & 18 & 17 & 16 & 15 & 14 & 13 & 12 & 11 & 10 & 9 & 8 & 7 & 6 & 5 & 4 & 3 & 2 & 1 & 0\\
\hline
x & x & $a_6$ & $a_5$ & $a_4$ & $a_3$ & $a_2$ & $a_1$ & $a_0$ & x & x & x & x & $b_8$ & $b_7$ & $b_6$ & $b_5$ & $b_4$ & $b_3$ & $b_2$ & $b_1$ & $b_0$ & x & x & $c_4$ & $c_3$ & $c_2$ & $c_1$ & $c_0$ & x & x & x\\
\hline
\end{tabular}}
\end{table}

\begin{enumerate}[label={a)}, leftmargin=*]
\item Geben Sie die Wertebereiche (dezimal) für ganze Zahlen an, die in $a$, $b$, $c$  i) vorzeichenlos und ii) als Zweierkomplemente dargestellt werden können!
\item[b)] In Hexadezimaldarstellung sei $w = \text{\texttt{E0A86780}}$. Geben Sie die Dezimaldarstellung der Zahlen in den Bitfeldern $a$, $b$ und $c$ an, wenn sie i) vorzeichenlos und ii) als Zweierkomplemente interpretiert werden!
\item[c)] Geben Sie das 32-bit Wort $w$ in Hexadezimaldarstellung an, wenn in den Bitfeldern die Werte $a : 25$, $b : -1$ und $c = -4$ als Zweierkomplemente dargestellt sind und die übrigen Bits 0 sind!
\end{enumerate}

\newpage