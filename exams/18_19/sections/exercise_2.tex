\section*{Aufgabe 2 (10 Punkte)}

\begin{enumerate}[label={a)}, leftmargin=*]
\item Berechnen Sie für die zehn im dreistelligen Hexadezimalsystem dargestellten Zahlen gemäß untenstehender Tabelle die fünf Summen als dreistellige Hexadezimalziffernfolgen und geben Sie jeweils an, ob ein Carry und/oder ein Overflow auftritt!

\begin{table}[h]
\centering
\begin{tabular}{|cc|cc|cc|cc|cc|}
& \texttt{138} & & \texttt{538} & & \texttt{EDA} & & \texttt{ECE} & & \texttt{9FC}\\
\texttt{+} & \texttt{73D} & \texttt{+} & \texttt{8A2} & \texttt{+} & \texttt{FAC} & \texttt{+} & \texttt{753} & \texttt{+} & \texttt{A0F}\\
\hline
& & & & & & & & &\\
\end{tabular}
\end{table}

\item[b)] Interpretieren Sie die zehn dreistelligen Hexadezimalziffernfolgen der Tabelle aus Aufgabenteil a) als Zahlen in \textbf{16er-Komplementdarstellung} und ordnen Sie sie ihrem Wert nach aufsteigend!
\end{enumerate}