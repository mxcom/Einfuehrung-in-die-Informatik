\section*{Aufgabe 6 (10 Punkte)}

Gegeben sei eine Boolesche Algebra $(B, +, \cdot)$ mit $B = \{0, 1\}$ und die Funktion $f : B^4 \rightarrow B$ als Boolescher Ausdruck, in dem \textbf{alle} ihre Primimplikanten disjunktiv verknüpft sind:

$$
f(x_3, x_2, x_1, x_0) = x'_3 x'_2 + x'_2 x'_1 + x'_2 x_0 + x_3x_0 + x_3 x_2 x_1 + x'_3 x_1 x'_0 + x_2 x_1 x'_0
$$

\begin{enumerate}[label={a)}, leftmargin=*]
\item Welche der Primimplikanten sind essentiell? Werden für die Darstellung der Funktion auch nicht essentielle Primimplikanten benötigt? Erstellen Sie eine Primimplikantentabelle und konstruieren Sie mit ihr einen möglichst einfachen Booleschen Ausdruck (Disjunktion von Primimplikanten) für die Funktion!
\item[b)] Die Funktion $f(x_3,x_2,x_1,x_0)$ soll jetzt so geändert werden, dass auch der Punkt $(1, 0, 1, 0) \in B^4$ auf $1 \in B$ abgebildet wird. Geben Sie einen möglichst einfachen Booleschen Ausdruck (Disjunktion von Primimplikanten) für die geänderte Funktion an!
\end{enumerate}