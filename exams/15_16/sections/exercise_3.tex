\section*{Aufgabe 3 (10 Punkte)}

\begin{enumerate}[label={a)}, leftmargin=*]
\item Geben sie für die Zahlen i) \texttt{25.375} und ii) \texttt{45.45} jeweils die binäre Gleitkommadarstellung an, ggfs. unter Kennzeichung der Periode!
\item[b)] Die beiden 32-bit Wörter i) \texttt{0x42820000} und ii) \texttt{0xBEC00000} sollen interpretiert werden als Gleitkommazahlen im IEEE 32-bit Format:

\begin{table}[h]
\centering
\resizebox{\textwidth}{!}{\begin{tabular}{|c|c|c|c|c|c|c|c|c|c|c|c|c|c|c|c|c|c|c|c|c|c|c|c|c|c|c|c|c|c|c|c|}
\hline
31 & 30 & 29 & 28 & 27 & 26 & 25 & 24 & 23 & 22 & 21 & 20 & 19 & 18 & 17 & 16 & 15 & 14 & 13 & 12 & 11 & 10 & 9 & 8 & 7 & 6 & 5 & 4 & 3 & 2 & 1 & 0\\
\hline
\texttt{BZ} & \multicolumn{8}{|c|}{\texttt{Exponent + 0x7F}} & \multicolumn{23}{|c|}{\texttt{Mantisse ohne "Hiden Bit"}}\\
\hline
\end{tabular}}
\end{table}

Bestimmen Sie jeweils die Darstellung der Zahl in der Form \texttt{VZ Mantisse * 2}$^\text{\texttt{Exponent}}$ und als dezimale Gleitkommazahl!
\end{enumerate}

