\section*{Aufgabe 1 (10 Punkte)}

\begin{enumerate}[label={a)}, leftmargin=*]
\item Führen Sie die fünf 16-bit Additionen gemäß untenstehenden Tabelle aus und geben Sie die Summen in Hexadezimaldarstellung an!

\begin{table}[h]
\centering
\begin{tabular}{|cc|cc|cc|cc|cc|}
& \texttt{EF30} & & \texttt{4215} & & \texttt{038A} & & \texttt{9FC2} & & \texttt{8000}\\
\texttt{+} & \texttt{FC05} & \texttt{+} & \texttt{5CCF} & \texttt{+} & \texttt{761B} & \texttt{+} & \texttt{9478} & \texttt{+} & \texttt{38F7}\\
\hline
& & & & & & & & &\\
\end{tabular}
\end{table}

\item[b)] Geben Sie für jede der fünf Summen aus Aufgabenteil a) an, ob ein Overflow und/oder ein Carry auftritt!
\item[c)] Interpretieren Sie die zehn 16-bit-Zahlen in obenstehender Tabelle als Zweierkomplemente und ordnen Sie diese der Größe nach (kleinste Zahl zuerst)!
\item[d)] Erweitern Sie die zehn Zahlen aus obenstehender Tabelle um vier bit auf 20 bit, so dass ihre Werte bei Interpretation als Zweierkomplement gleich bleiben (signed extension)!
\end{enumerate}